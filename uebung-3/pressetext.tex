\documentclass[%
paper=a4,                       % alle weiteren Papierformat einstellbar
%landscape,                     % Querformat
fontsize=11pt,                  % Schriftgröße (12pt, 11pt (Standard))
%BCOR1cm,                       % Bindekorrektur, bspw. 1 cm
DIV=calc,                       % führt die Satzspiegelberechnung neu aus
                                % s. scrguide 2.4
%twoside,                       % Doppelseiten
%twocolumn,                     % zweispaltiger Satz
parskip=half,                   % Absatzformatierung s. scrguide 3.1
%headsepline,                   % Trennline zum Seitenkopf  
%footsepline,                   % Trennline zum Seitenfuß
%titlepage,                     % Titelei auf eigener Seite
headings=big,               % Überschriften etwas kleiner (smallheadings)
%idxtotoc,                      % Index im Inhaltsverzeichnis
%liststotoc,                    % Abb.- und Tab.verzeichnis im Inhalt
bibliography=totoc,             % Literaturverzeichnis im Inhalt
%abstracton,                    % Überschrift über der Zusammenfassungan    
%leqno,                         % Nummerierung von Gleichungen links
%fleqn,                         % Ausgabe von Gleichungen linksbündig
%draft                          % überlangen Zeilen in Ausgabe gekennzeichnet
]
{scrartcl}

\usepackage[pdftex]{graphicx} 
\usepackage[english]{babel} 
\usepackage[utf8]{inputenc}     
\usepackage[T1]{fontenc}                                
\usepackage{lmodern}
\usepackage{color}

% Links im PDF
\usepackage{hyperref}
\definecolor{LinkColor}{rgb}{0,0,0.5}
\hypersetup{
    colorlinks=true,
    linkcolor=LinkColor,
    citecolor=LinkColor,
    filecolor=LinkColor,
    menucolor=LinkColor,
    urlcolor=LinkColor}
    
%Kopf u. Fußzeilen
\usepackage{scrpage2}
\pagestyle{scrheadings}
\clearscrheadfoot
%\chead{}
%\ofoot{}
\cfoot{\pagemark}

\usepackage{mdwlist}
\usepackage{csquotes}

\newcommand{\zitat}[1]{\emph{\enquote{#1}}}

% Neues bibliographie paket
% style=apa
\usepackage[isbn=false,url=false, doi=false, backend=biber, style=numeric]{biblatex}
\DeclareLanguageMapping{american}{american-apa}

\KOMAoptions{DIV=last}


\begin{document} 

\section*{Agile Softwareentwicklungsmethoden im Vormarsch}

\subsection*{Schnell -- Flexibel -- Kreativ -- Innovativ:  Eigenschaften eines
modernen Softwareentwicklers um am Markt ganz vorn dabei zu sein}

Um innovative Produkte entwickeln und gleichzeitig auf Veränderungen des
Marktes und auf Kundenwünsche schnell reagieren zu können bedarf es den
Einsatz agiler Softwareentwicklungsmethoden. Vorteil dieser Methoden im
Gegensatz zu traditionellen schwergewichtigen Methoden ist, dass sie kurze
Entwicklungszyklen beinhalten ohne jedoch die wichtigen Phasen wie Planung,
Analyse, Umsetzung und Test zu vernachlässigen. Darüber hinaus sind die
Entwicklungsteams bei dieser Methode eher klein (5--9 Mitglieder),
selbstorganisiert, jedes Mitglied in allen Bereichen mit eingebunden und die
Kommunikation erfolgt direkt, sodass eine lange aufwändige Dokumentation
obsolet wird. Wird ein Zyklus beendet, wird das Produkt den Stakeholdern zur
Diskussion vorgelegt um erfolgreich weitere Entwicklungsprozesse einzuleiten.
Prominente Beispiele für agile Ansätze sind: \emph{Extreme Programming} und
\emph{Scrum}. Bei beiden Methoden liegen die Hauptaugenmerke in der der
direkten Kooperation zwischen Manager, Klienten und Entwicklern. Kurze
Entwicklungszyklen, inkrementelle Planung, kontinuierliches Feedback,
Kommunikation und evolutionäres Design zeichnen diese Prozesse aus. Der
Austausch von Wissen und Errungenschaften steht dabei ebenso im Vordergrund um
einen optimierten, schnellen und effizienten Prozess zu ermöglichen.

Eines ist klar, um in Zukunft den schnell wechselnden Anforderungen am
Softwaremarkt gewachsen zu sein, ist die Anwendung des agilen Ansatzes
unerlässlich.

Aber wenn man ehrlich ist, ist es doch schön Mitglied eines kleinen kreativen
Softwareteams zu sein um innovativ Ideen vorantreiben zu können anstatt sich
dauernd durch Dokumentationen zu quälen.

\end{document}
