\documentclass[%
paper=a4,                       % alle weiteren Papierformat einstellbar
%landscape,                     % Querformat
fontsize=11pt,                  % Schriftgröße (12pt, 11pt (Standard))
%BCOR1cm,                       % Bindekorrektur, bspw. 1 cm
DIV=calc,                       % führt die Satzspiegelberechnung neu aus
                                % s. scrguide 2.4
%twoside,                       % Doppelseiten
%twocolumn,                     % zweispaltiger Satz
parskip=half,                   % Absatzformatierung s. scrguide 3.1
%headsepline,                   % Trennline zum Seitenkopf  
%footsepline,                   % Trennline zum Seitenfuß
titlepage,                      % Titelei auf eigener Seite
headings=big,               % Überschriften etwas kleiner (smallheadings)
%idxtotoc,                      % Index im Inhaltsverzeichnis
%liststotoc,                    % Abb.- und Tab.verzeichnis im Inhalt
bibliography=totoc,             % Literaturverzeichnis im Inhalt
%abstracton,                    % Überschrift über der Zusammenfassungan    
%leqno,                         % Nummerierung von Gleichungen links
%fleqn,                         % Ausgabe von Gleichungen linksbündig
%draft                          % überlangen Zeilen in Ausgabe gekennzeichnet
]
{scrartcl}

\usepackage[pdftex]{graphicx} 
\usepackage[ngerman]{babel} 
\usepackage[utf8]{inputenc}     
\usepackage[T1]{fontenc}                                
\usepackage{lmodern}
\usepackage{color}
\usepackage{booktabs}
\usepackage{array}

% Links im PDF
\usepackage{hyperref}
\definecolor{LinkColor}{rgb}{0,0,0.5}
\hypersetup{
    colorlinks=true,
    linkcolor=LinkColor,
    citecolor=LinkColor,
    filecolor=LinkColor,
    menucolor=LinkColor,
    urlcolor=LinkColor}
    
%Kopf u. Fußzeilen
\usepackage{scrpage2}
\pagestyle{scrheadings}
\clearscrheadfoot
%\chead{}
%\ofoot{}
\cfoot{\pagemark}

\usepackage{mdwlist}
\usepackage{csquotes}

% Neues bibliographie paket
% style=apa
\usepackage[isbn=false,url=false, doi=false, backend=biber, style=numeric]{biblatex}
%\DeclareLanguageMapping{american}{american-apa}

\usepackage{pdfpages}

\KOMAoptions{DIV=last}

\begin{document}

\begin{titlepage}
\sffamily

{ \Large Forschungsmethoden WS 2012} \\[2cm]
    
{ \Huge \centering \bfseries Übung 5: Zusammenfassung \\[1.5cm] }

%{ \LARGE \centering \bfseries \varpressetext \\[1.5cm] }

%{
%    \begin{flushright}
%    { \large
%       { \bfseries Bewertete Gruppe:~\vargruppe \\ }
%       \varmitglieder
%    }
%
%    \end{flushright}
%}

\vfill

{ \large
   { \bfseries Gruppe: 8 \\ }
   Bernhard Fleck \\
   Rafael Konik \\
   Stephan Matiasch \\
   Harald Watzke \\
}
\end{titlepage}

\section{Zusammenfassung}

\begin{table}[!ht]
   \begin{center}
	\begin{tabular}{p{6cm}lm{3cm}}
    \toprule
	\textbf{Was haben wir gelernt?} & \textbf{Wobei?} & \textbf{Wieviel haben wir davon profitiert? 0\%~--~100\%} \tabularnewline
    \midrule
	 Literatur Recherche & Übung 1 & 70\% \tabularnewline
     \addlinespace
 	 Literatur Zitieren & Übung 1 \& 3 & 65\% \tabularnewline
     \addlinespace
	 Implementieren & Übung 2 & 15\% \tabularnewline
     \addlinespace
	 Benchmarking & Übung 2 & 35\% \tabularnewline
     \addlinespace
	 Interviews, Befragungen & Übung 4 & 75\% \tabularnewline
     \addlinespace
	 Technisches all\-ge\-mein\-ver\-ständ\-lich darstellen & Übung 3 & 50\% \tabularnewline
     \addlinespace
	 Schreiben eines Technical Paper & Übung 3 & 55\% \tabularnewline
     \addlinespace
	 Präsentationstechnik & Übung 1 \% 4 & 60\% \tabularnewline
     \addlinespace
	 Wissenschaftliches Arbeiten & bla & 75\% \tabularnewline

     \bottomrule
	\end{tabular}
   \end{center}
\end{table}

\section{Gesamtaufwand der Gruppe}

Der Gesamtaufwand der Gruppe betrug etwa 290 Stunden.\\ Dabei entfiel mehr auf Übung 1 und Übung 2, aufgrund des hohen Stundenaufwands für das Einlesen in die Diplomarbeiten bei Übung 1 und der Implementierungsaufwand bei Übung 2.

\end{document}
