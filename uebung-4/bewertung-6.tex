\documentclass[%
paper=a4,						% alle weiteren Papierformat einstellbar
%landscape,						% Querformat
fontsize=11pt,					% Schriftgröße (12pt, 11pt (Standard))
%BCOR1cm,						% Bindekorrektur, bspw. 1 cm
DIV=calc,						% führt die Satzspiegelberechnung neu aus
								% s. scrguide 2.4
%twoside,						% Doppelseiten
%twocolumn,						% zweispaltiger Satz
parskip=half,					% Absatzformatierung s. scrguide 3.1
%headsepline,					% Trennline zum Seitenkopf	
%footsepline,					% Trennline zum Seitenfuß
titlepage,						% Titelei auf eigener Seite
headings=big,				% Überschriften etwas kleiner (smallheadings)
%idxtotoc,						% Index im Inhaltsverzeichnis
%liststotoc,					% Abb.- und Tab.verzeichnis im Inhalt
bibliography=totoc,				% Literaturverzeichnis im Inhalt
%abstracton,					% Überschrift über der Zusammenfassungan	
%leqno,   						% Nummerierung von Gleichungen links
%fleqn,							% Ausgabe von Gleichungen linksbündig
%draft							% überlangen Zeilen in Ausgabe gekennzeichnet
]
{scrartcl}

\usepackage[pdftex]{graphicx} 
\usepackage[ngerman]{babel}	
\usepackage[utf8]{inputenc}		
\usepackage[T1]{fontenc}								
\usepackage{lmodern}
\usepackage{color}
\usepackage{booktabs}
\usepackage{array}

% Links im PDF
\usepackage{hyperref}
\definecolor{LinkColor}{rgb}{0,0,0.5}
\hypersetup{
	colorlinks=true,
	linkcolor=LinkColor,
	citecolor=LinkColor,
	filecolor=LinkColor,
	menucolor=LinkColor,
	urlcolor=LinkColor}
	
%Kopf u. Fußzeilen
\usepackage{scrpage2}
\pagestyle{scrheadings}
\clearscrheadfoot
%\chead{}
%\ofoot{}
\cfoot{\pagemark}

\usepackage{mdwlist}
\usepackage{csquotes}

% Neues bibliographie paket
% style=apa
\usepackage[isbn=false,url=false, doi=false, backend=biber, style=numeric]{biblatex}
%\DeclareLanguageMapping{american}{american-apa}

\usepackage{pdfpages}

\def \varpressetext {<Pressetext angeben>}
\newcommand{\pressetext}[1]{\def \varpressetext {#1}}

\def \vargruppe {<Gruppennummer angeben>}
\newcommand{\gruppe}[1]{\def \vargruppe {#1}}

\def \varmitglieder {<Gruppenmitglieder angeben>}
\newcommand{\mitglieder}[1]{\def \varmitglieder {#1}}

\KOMAoptions{DIV=last}


\pressetext{Location Awareness}
\gruppe{6}
\mitglieder{Michael Habiger \\ Johannes Hartl \\ Thomas Teufel \\ Patrick Wolf}

\begin{document} 

%!TEX root = bewertung-5.tex

\begin{titlepage}
\sffamily

{ \Large Forschungsmethoden WS 2012} \\[2cm]
    
{ \Huge \centering \bfseries Übung 4: Studienzusammenfassung \\[1.5cm] }

{ \LARGE \centering \bfseries \varpressetext \\[1.5cm] }

{
    \begin{flushright}
    { \large
       { \bfseries Bewertete Gruppe:~\vargruppe \\ }
       \varmitglieder
    }

    \end{flushright}
}

\vfill

{ \large
   { \bfseries Bewertende Gruppe: 8 \\ }
   Bernhard Fleck \\
   Rafael Konik \\
   Stephan Matiasch \\
   Harald Watzke \\
}
\end{titlepage}


\includepdf[pages=-]{fragebogen-\vargruppe}

\section{Auswertung von \enquote{\varpressetext}}

Eine Aufschlüsselung der Bewertungsskala des Fragebogens welche fortan für die
Kodierung der angekreuzten Antworten verwendet wird ist
Tabelle~\ref{tab:skala} zu entnehmen.

\begin{table}[!ht]
    \caption{Aufschlüsselung der Bewertungsskala}
    \label{tab:skala}
    \begin{center}
        \begin{tabular}{lc}
        \toprule
        \textbf{Bezeichnung} & \textbf{Skalenwert} \\
        \midrule
        Trifft zu & 1 \\
        Trifft eher zu & 2 \\
        Weder noch & 3 \\
        Trifft eher nicht zu & 4 \\
        Trifft nicht zu & 5 \\
        \bottomrule
        \end{tabular}
    \end{center}
\end{table}

\subsection{Auswertung der allgemeinen \enquote{meta}-Bewertung}

Die folgenden Aussagen lassen sich aufgrund der Bewertung treffen:

\begin{itemize}
    \item der Pressetext hat sehr gut gefallen
    \item der Pressetext ist sehr gut geschrieben
    \item der Pressetext einhält keine ungeklärte Begriffe
    \item der Pressetext ist sehr nützlich
    \item der \emph{Unique Selling Point} konnte mittelmäßig dargelegt werden
    \item der Pressetext ist sehr gut geschrieben
    \item das Thema des Pressetextes ist eher interessant
    \item die Problemstellung wurde eher gut dargelegt
\end{itemize}

\begin{table}[!ht]
    \begin{center}
        \begin{tabular}{lcc}
        \toprule
        \textbf{Aussage} & \textbf{kodierte Antworten} & \textbf{Median} \\
        \midrule
        Der Artikel hat mir gut gefallen & 2, 1, 1 & 1 \\
        Der Artikel ist gut verständlich & 2, 1, 1 & 1 \\
        Der Artikel enthält viele unerklärte Begriffe & 5, 5, 5 & 5 \\
        Der Artikel ist nützlich & 1, 1, 1 & 1 \\
        Der \emph{Unique Selling Point} wurde gut dargelegt & 2, 4, 3 & 3 \\
        Der Artikel ist gut geschrieben & 3, 1, 1 & 1 \\
        Das Thema des Artikels ist interessant & 2, 2, 1 & 2 \\
        Die Problemstellung wird gut dargelegt & 2, 2, 1 & 2 \\
        \bottomrule
        \end{tabular}
    \end{center}
\end{table}

Invertiert man die einzige negative Aussage \enquote{\emph{Der Artikel enthält
viele unerklärte Begriffe}} und mittelt über alle Aussagen hinweg, erhält man,
unter Verwendung der Skala: \enquote{gut gefallen}, \enquote{eher gut
gefallen}, \enquote{weder noch}, \enquote{eher nicht gefallen} und
\enquote{nicht gefallen}, den folgenden Median: 1--2.
Der Pressetext hat also insgesamt \enquote{gut gefallen} bis \enquote{eher gut gefallen}.

Die Übereinstimmung der Probanden untereinander was die Bewertung des
allgemeinen Teils angeht kann als \emph{gut} bezeichnet werden.


\subsection{Auswertung zum Verständnis des Inhaltes}

\begin{table}[!ht]
    \begin{center}
        \begin{tabular}{p{8cm}cc}
        \toprule
        \textbf{Aussage} & \textbf{kodierte Antworten} & \textbf{Median} \\
        \midrule
        Standortinformationen werden häufig verwendet & 1, 1, 1 & 1 \\
        \addlinespace
        Standortinformationen können bedenkenlos veröffentlich werden  & 5, 5, 5 & 5 \\
        \bottomrule
        \end{tabular}
    \end{center}
\end{table}

Die Auswertung zeigt, dass der Inhalt des Pressetextes von allen Probanden
\enquote{sehr gut verstanden} wurde. Die Übereinstimmung der Probanden
untereinander was die Bewertung des inhaltlichen Teils angeht kann als
\emph{sehr gut} bezeichnet werden.

\section{Zusammenfassung aller Studien} % (fold)
\label{sec:zusammenfassung_aller_studien}

% section zusammenfassung_aller_studien (end)

\includepdf[pages=-]{bewertungen/bewertung_\vargruppe_hw}
\includepdf[pages=-]{bewertungen/bewertung_\vargruppe_rk}
\includepdf[pages=-]{bewertungen/bewertung_\vargruppe_sm}

\end{document}
