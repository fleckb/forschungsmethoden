%!TEX root = report.tex

% Erste Arbeit
%
\section*{\thesisheading} % (fold)

\thesis{Titel}{Security mechanisms for low-end embedded systems}
\thesis{AutorIn}{Thomas Flanitzer}
\thesis{Betreuung}{Univ.-Prof. Dipl.-Ing. Dr. Wolfgang Kastner \& Mag. Dipl.-Ing. Fritz Praus}
\thesis{Preis}{Frequentis Förderpreis}

% 3-5 Sätze zu jedem der folg. Punkte jeder Diplomarbeit

\minisec{Themengebiet} Das Themengebiet umfasst die Security-Probleme von \enquote{low-end embedded systems}. 
Insbesondere wird hier auf Heim- und Gebäudeautomationssysteme eingegangen. Um die Sicherheit der low-end ES zu 
verbessern wird eine mögliche Lösung implementiert und getestet. [Kap.~1]{Flanitzer2008}

\minisec{Problemstellung} Das Ziel der Arbeit ist ein Security-Mechanismus für low-end ES, welcher das System 
vor Angriffen schützt. Gleichzeitig sollen trotzdem noch ungeprüfte und daher eventuell fehlerhafte Prgramme 
auf dem System ausgeführt werden können. Trotz der begrenzten Ressourcen soll ein möglichst guter Mechanismus 
entwickelt und getestet werden. [Kap.~1]{Flanitzer2008}

\minisec{Verwendete Methoden} Als Forschungsmethoden kommen Literatur, Recherche \& Zitieren, Modellierung, 
Implementierung, Benchmarking und kontrollierte Experimente zum Einsatz. Bei der Implementierung wird vorhandene
Sicherheitssoftware erweitert und modifiziert. Diese erweiterte Software wird dann in einer bereitgestellten Testumgebung
getestet und die Ergebnisse des Experimentes protokolliert. [Kap.~1,5,6,7]{Flanitzer2008}

\minisec{Ergebnis} Der entwickelte Security-Mechanismus für low-end ES erlaubt die Kontrolle über das Ausführen von
Anwenderprogrammen. Die Stabilität des Security-Mechanismus ist zufriedenstellend. Es gab hier nur ein paar Probleme beim
Empfangen von Nachrichten, wobei der Grund dafür gefunden, aber nicht behoben wurde. Die Performance ist nicht allzu gut,
während die angestrebte Flexibilität allerdings erreicht wurde. [Kap.~6,7]{Flanitzer2008}


% Zweite Arbeit
%
\section*{\thesisheading} % (fold)

\thesis{Titel}{<Titel>}
\thesis{AutorIn}{<AutorIn>}
\thesis{Betreuung}{<Betreuung>}
\thesis{Preis}{<ja \& welcher / nein>}

% 3-5 Sätze zu jedem der folg. Punkte jeder Diplomarbeit

\minisec{Themengebiet}

\minisec{Problemstellung}

\minisec{Verwendete Methoden}

\minisec{Ergebnis}


% Dritte Arbeit
%
\section*{\thesisheading} % (fold)

\thesis{Titel}{<Titel>}
\thesis{AutorIn}{<AutorIn>}
\thesis{Betreuung}{<Betreuung>}
\thesis{Preis}{<ja \& welcher / nein>}

% 3-5 Sätze zu jedem der folg. Punkte jeder Diplomarbeit

\minisec{Themengebiet}

\minisec{Problemstellung}

\minisec{Verwendete Methoden}

\minisec{Ergebnis}


% section section_name (end)
