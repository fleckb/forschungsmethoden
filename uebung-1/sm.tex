%!TEX root = report.tex

% Erste Arbeit
%
\section*{\thesisheading} % (fold)

\thesis{Titel}{Security mechanisms for low-end embedded systems}
\thesis{AutorIn}{Thomas Flanitzer}
\thesis{Betreuung}{Univ.-Prof. Dipl.-Ing. Dr. Wolfgang Kastner \& Mag. Dipl.-Ing. Fritz Praus}
\thesis{Preis}{Frequentis Förderpreis}

% 3-5 Sätze zu jedem der folg. Punkte jeder Diplomarbeit

\minisec{Themengebiet} Das Themengebiet umfasst die Security-Probleme von \enquote{low-end embedded systems}. 
Insbesondere wird hier auf Heim- und Gebäudeautomationssysteme eingegangen. Um die Sicherheit der low-end ES zu 
verbessern wird eine mögliche Lösung implementiert und getestet. \cite[Kap.~1]{Flanitzer2008}

\minisec{Problemstellung} Das Ziel der Arbeit ist ein Security-Mechanismus für low-end ES, welcher das System 
vor Angriffen schützt. Gleichzeitig sollen trotzdem noch ungeprüfte und daher eventuell fehlerhafte Prgramme 
auf dem System ausgeführt werden können. Trotz der begrenzten Ressourcen soll ein möglichst guter Mechanismus 
entwickelt und getestet werden. \cite[Kap.~1]{Flanitzer2008}

\minisec{Verwendete Methoden} Als Forschungsmethoden kommen Literatur, Recherche \& Zitieren, Modellierung, 
Implementierung, Benchmarking und kontrollierte Experimente zum Einsatz. Bei der Implementierung wird vorhandene
Sicherheitssoftware erweitert und modifiziert. Diese erweiterte Software wird dann in einer bereitgestellten Testumgebung
getestet und die Ergebnisse des Experimentes protokolliert. \cite[Kap.~1,5,6,7]{Flanitzer2008}

\minisec{Ergebnis} Der entwickelte Security-Mechanismus für low-end ES erlaubt die Kontrolle über das Ausführen von
Anwenderprogrammen. Die Stabilität des Security-Mechanismus ist zufriedenstellend. Es gab hier nur ein paar Probleme beim
Empfangen von Nachrichten, wobei der Grund dafür gefunden, aber nicht behoben wurde. Die Performance ist nicht allzu gut,
während die angestrebte Flexibilität allerdings erreicht wurde. \cite[Kap.~6,7]{Flanitzer2008}


% Zweite Arbeit
%
\section*{\thesisheading} % (fold)

\thesis{Titel}{Analyse, Auswertung und Visualisierung umfangreicher textueller Daten mit dem Schwerpunkt der elektronischen Bürgerzufriedenheitserhebung als Aspekt des e-Government}
\thesis{AutorIn}{Elham Hedayati-Rad}
\thesis{Betreuung}{Univ.-Prof. Dipl.-Ing. Dr. Thomas Grechenig}
\thesis{Preis}{nein}

% 3-5 Sätze zu jedem der folg. Punkte jeder Diplomarbeit

\minisec{Themengebiet} Das Themengebiet umfasst Meinungsforschung und deren Durchführung. Als Hauptthema der Arbeit wird
auf die Meinungsforschung im e-Government eingegangen. Es wird der Prozess der Datenerhebung bis hin zur Auswertung
der Daten beschrieben und umgesetzt. \cite[Kap.~1]{Hedayati-Rad2008}

\minisec{Problemstellung} Das Ziel der Arbeit ist es den Prozess der Datenerhebung und Datenanalyse zu erläutern und
hierfür werden Algorithmen zur Analyse von unstrukturierten Texten vorgestellt. Außerdem wird ein Überblick über die
empirische Sozialforschung und Meinungsforschung gegeben. Schlussendlich wird noch eine Bürgerbefragung durchgeführt 
und ausgewertet. \cite[Kap.~1]{Hedayati-Rad2008}

\minisec{Verwendete Methoden} Als Forschungsmethoden werden Literatur, Recherche \& Zitieren und die Anwendung der
Algorithmen in der Praxis verwendet. Im Rahmen der Diplomarbeit wird eine Bürgerzufriedenheitsbefragung durchgeführt. 
Die dadurch erhaltenen Texte werden dann mit vorgestellten Algorithmen analysiert. \cite[Kap.~1,8]{Hedayati-Rad2008}

\minisec{Ergebnis} Das Ergebnis der Arbeit ist, dass automatische Algorithmen zur Analyse von unstrukturierten Texten 
nur unter äußerster Vorsicht zu verwenden ist. Durch die automatischen Algorithmen steigt die Gefahr fehlerhafte Daten
zu erheben. Außerdem ist es sehr wichtig, dass diese Algorithmen mehrere Sprachen unterstützen. \cite[Kap.~9]{Hedayati-Rad2008}


% Dritte Arbeit
%
\section*{\thesisheading} % (fold)

\thesis{Titel}{Identifikation und Analyse sicherheitsrelevanter Kriterien für Mobile Contactless Payment-Systeme zur Förderung der Akzeptanz beim Endanwender}
\thesis{AutorIn}{Yvonne Hren}
\thesis{Betreuung}{Univ.-Prof. Dipl.-Ing. Dr. techn. Thomas Grechenig \& Dipl.-Ing. Mag. Andreas Ehringfeld}
\thesis{Preis}{nein}

% 3-5 Sätze zu jedem der folg. Punkte jeder Diplomarbeit

\minisec{Themengebiet} Das Themengebiet umfasst die immer intensivere Nutzung der Möglichkeit mit einem Mobiltelefon
zu bezahlen. Diskussionsthema der Arbeit sind die sicherheitsrelevanten Kriterien für diese Möglichkeit. Ein großes Thema
ist die Benutzerfreundlichkeit und das Eingehen auf die Bedürfnisse der Mobiltelefonbesitzer. \cite[Kap.~1]{Hren2008}

\minisec{Problemstellung} Ein Ziel der Arbeit ist es die Systeme des Mobile-Payments in Österreich vorzustellen. Außerdem
sollen die Gefahren im Bezug auf das Mobile Contactless Payment erläutert werden. Schlussendlich sollen Mindestkriterien
für die Akzeptanzförderung beim Endanwender identifiziert werden. \cite[Kap.~1.1]{Hren2008}

\minisec{Verwendete Methoden} Als Forschungsmethoden werden Literatur, Recherche \& Zitieren und Studien \& Meinungsumfragen
verwendet. Um das Ziel der Akzeptanz beim Endanwender möglichst gut zu erfüllen werden Umfragen angeführt. Diese Umfragen
zeigen die für den Endanwender wichtigen Kriterien im Bezug auf die Sicherheit. \cite[Kap.~1,7]{Hren2008}

\minisec{Ergebnis} Das Ergebnis der Arbeit ist, dass die Sicherheit das wichtigste Kriterium beim Bezahlen mit dem 
Mobiltelefon darstellt. Allerdings spielen hierbei noch einige weitere Kriterien eine Rolle. Diese Kriterien wären zum
Beispiel die Vertraulichkeit der Daten, die Akzeptanzstellen, die Benutzerfreundlichkeit und die Kosten. Um beim 
Endanwender auf Akzeptanz zu stoßen müssen möglichst viele dieser Kriterien erfüllt werden. \cite[Kap.~8]{Hren2008}


% section section_name (end)
