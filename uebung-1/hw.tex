%!TEX root = report.tex

% Erste Arbeit
%
\section*{\thesisheading} % (fold)

\thesis{Titel}{Design and Implementation of the JavaSpaces API Standard for XVSM}
\thesis{AutorIn}{Laszlo Keszthelyi}
\thesis{Betreuung}{Eva Kühn und Richard Mordinyi}
\thesis{Preis}{Nein}

% 3-5 Sätze zu jedem der folg. Punkte jeder Diplomarbeit
\minisec{Themengebiet}
Diese Diplomarbeit ist im Themengebiet Middleware, eine Softwarearchitektur angesiedelt. Im speziellen geht es um eine neue Version der Space Based Middleware, eXtensible Virtual Shared Memory (XVSM), welche von der Space Based Computing Group der TU Wien entwickelt wurde. Diese dienen der Kollaboration zwischen verteilten Prozessen mittels eines virtuellen Raums. Eine ältere Variante einer solchen Architektur, JavaSpaces und damit verbunden JINI, ist ebenfalls relevant für diese Arbeit, genauso wie Linda und Tuplespaces.

\minisec{Problemstellung}
Das Hauptthema der Arbeit ist die Entwicklung und Beschreibung des JavaSpaces API Standard für MozartSpaces, die XVSM Java Referenzimplementation. Eine High-Level Java API, eine JavaScript API sowie eine JMS API waren damals bereits für XVSM vorhanden. Weiters wird versucht den bereits in MozartSpaces enthaltenen LindaCoordinator, eine Implementation der Linda Koordinationssprache, zu verbessern.
 
\minisec{Verwendete Methoden}
Die verwendeten Methoden waren in erster Linie Literaturrecherche \& Zitieren, Implementieren sowie Benchmarking. Literaturrecherche \& Zitieren unter anderem da eine grundsätzliche Erklärung der Grundlagen zu Themen wie der Linda Koordinationssprache, Tuplespaces, Jini, JavaSpaces und XVSM nötig waren. Implementiert wurde natürlich die JavaSpaces API für XVSM (JAXS) und die Überarbeitung des LindaCoordinators. Weiters wurden Benchmarks sowohl für die überarbeitete Version des LindaCoordinators wie auch für JAXS selbst durchgeführt.

\minisec{Ergebnis}
Aus der Arbeit resultierten primär der überarbeitete LindaCoordinator sowie die JavaSpaces API für MozartSpaces. Diese API erlaubt es für JavaSpaces entwickelte Systeme und Anwendungen ohne Änderung auch MozartSpaces zu verwenden. Die erstellten Benchmarks beweisen, dass das entwickelte JavaSpaces Frontend nicht viel langsamer, in manchen Fällen sogar schneller, als bestehende JavaSpaces Implementationen ist.

% Zweite Arbeit
%
\section*{\thesisheading} % (fold)

\thesis{Titel}{Design and Implementation of XcoSpaces, the .Net Reference Implementation of XVSM - Core Architecture and Aspects}
\thesis{AutorIn}{Thomas Scheller}
\thesis{Betreuung}{Eva Kühn}
\thesis{Preis}{Nein}

% 3-5 Sätze zu jedem der folg. Punkte jeder Diplomarbeit
\minisec{Themengebiet}
Diese Diplomarbeit ist im Themengebiet Middleware, eine Softwarearchitektur angesiedelt. Im speziellen geht es um eine neue Version der Space Based Middleware, eXtensible Virtual Shared Memory (XVSM), welche von der Space Based Computing Group der TU Wien entwickelt wurde. Diese dienen der Kollaboration zwischen verteilten Prozessen mittels eines virtuellen Raums.

\minisec{Problemstellung}
Das Ziel dieser Diplomarbeit war die Entwicklung einer .NET Referenzimplementation für XVSM, XcoSpaces. In dieser Arbeit entwickelt Scheller XcoSpaces, im speziellen die Kernarchitektur sowie Aspekte. Es wird ebenfalls eine kurze Einführung in XVSM geliefert.

\minisec{Verwendete Methoden}
Die verwendeten Methoden waren Literaturrecherche \& Zitieren sowie Implementieren. Diese Arbeit wurde im Zusammenspiel mit den Diplomarbeiten anderer Studenten durchgeführt, auf diese wird immer wieder verwiesen. Ebenso wird bei der Beschreibung der Architektur der Software auf vorhergehende Arbeiten verwiesen.

\minisec{Ergebnis}
Es wird eine prinzipielle Darstellung und Erklärung des XVSM Kern geliefert. Weiters sind natürlich der Kern von XcoSpaces sowie Unterstützung für Aspekte, ein Mechanismus zur Erweiterung von XcoSpaces und eine der Stärken der XVSM-Technologie, Ergebnisse dieser Arbeit. Scheller liefert außerdem ein System zur Klassifizierung von System die nach dem Shared Data Spaces Paradigm arbeiten und gibt auch eine kurze Übersicht über einiger dieser Systeme.

% Dritte Arbeit
%
\section*{\thesisheading} % (fold)

\thesis{Titel}{Solution Methods for the Social Golfer Problem}
\thesis{AutorIn}{Markus Triska}
\thesis{Betreuung}{Nysret Musliu}
\thesis{Preis}{Ja}\begin{itemize}
\item Distinguished Young Alumnus Award der Fakultät für Informatik
\item Microsoft Förderpreis
\end{itemize}

% 3-5 Sätze zu jedem der folg. Punkte jeder Diplomarbeit

\minisec{Themengebiet}
Diese Diplomarbeit ist aufgrund des behandelten Social Golfer Problems im Bereich der kombinatorischen Optimisierungsprobleme angesiedelt. Es gibt Überschneidungen mit Geometrie und Graphentheorie. Es werden auch die Themenbereiche Codierung, Verschlüsselung und Taskbearbeitung als verbunden erwähnt. Aufgrund der besprochenen Lösungen und Optimierungen sind auch Formale Methoden, Diskrete Mathematik sowie Scheduling von Interesse.

\minisec{Problemstellung}


\minisec{Verwendete Methoden}

\minisec{Ergebnis}


% section section_name (end)