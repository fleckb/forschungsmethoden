%!TEX root = report.tex

% Erste Arbeit
%
\section*{\thesisheading}

\thesis{Titel}{Biomechanische Schlaganalyse im Kyokushin-Karate}
\thesis{AutorIn}{Melanie Fraunschiel}
\thesis{Betreuung}{Winfried Mayr}
\thesis{Preis}{Best Poster Award der Fakultät für Informatik}

% 3-5 Sätze zu jedem der folg. Punkte jeder Diplomarbeit

\minisec{Themengebiet}

\minisec{Problemstellung}

\minisec{Verwendete Methoden}

\minisec{Ergebnis}
Referenzbeispiel mit Seitenangabe:~\cite[23]{Fraunschiel2008}.

% Zweite Arbeit
%
\section*{\thesisheading}

\thesis{Titel}{Evaluating Object-Oriented Software Metrics for Source Code
               Change Analysis -- A Study on Open Source Projects}
\thesis{AutorIn}{Andreas Mauczka}
\thesis{Betreuung}{Thomas Grechenig und Mario Bernhart}
\thesis{Preis}{nein}

% 3-5 Sätze zu jedem der folg. Punkte jeder Diplomarbeit

\minisec{Themengebiet}

\minisec{Problemstellung}

\minisec{Verwendete Methoden}

\minisec{Ergebnis}


% Dritte Arbeit
%
\section*{\thesisheading}

\thesis{Titel}{J3DVN -- A Generic Framework for 3D Software Visualization}
\thesis{AutorIn}{Florian Breier}
\thesis{Betreuung}{Harald Gall und Jacek Ratzinger}
\thesis{Preis}{nein}

% 3-5 Sätze zu jedem der folg. Punkte jeder Diplomarbeit

\minisec{Themengebiet}

\minisec{Problemstellung}

\minisec{Verwendete Methoden}

\minisec{Ergebnis}


% section section_name (end)