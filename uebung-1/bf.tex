%!TEX root = report.tex

% Erste Arbeit
%
\section*{\thesisheading}

\thesis{Titel}{Biomechanische Schlaganalyse im Kyokushin-Karate}
\thesis{AutorIn}{Melanie Fraunschiel}
\thesis{Betreuung}{Winfried Mayr}
\thesis{Preis}{Best Poster Award der Fakultät für Informatik}

% 3-5 Sätze zu jedem der folg. Punkte jeder Diplomarbeit

\minisec{Themengebiet} Das Themengebiet umfasst Vollkontakt-Kampfsportarten.
Insbesondere wird auf Kyokushin-Karate eingegangen und biomechanisch
analysiert. Weiters wird auf vergleichbare Analysen im Boxsport verwiesen.

\minisec{Problemstellung} Das Ziel der Arbeit ist die Analyse von Bewegungen
sowie von Schlag- und Trittwirkung im Kyokushin-Karate. Weiters werden
verschiedene Schlag- und Tritttechniken untereinander verglichen
\cite[Kap.~4,~S.~6f]{Fraunschiel2008}.

\minisec{Verwendete Methoden} Als verwendete Methoden kommen Literatur
Recherche \& Zitieren sowie kontrollierte Experimente \& Messen zum Einsatz.
Bei den kontrollierten Experimenten werden mehrere Schlag- und Tritttechniken
von mehreren Probanden an einem Boxsack, sowie einem im Boden verankertem
Holzbrett durchgeführt. Für die Messung werden modernste Sensortechnik und
Hich-Speed Kameras verwendet \cite[Kap.~8,~S.~54ff]{Fraunschiel2008}.

\minisec{Ergebnis}
Die folgenden Aussagen können als Ergebnisse der Arbeit bestätigt getätigt
werden:
\begin{itemize*}
    \item Beleibte Kämpfer stecken Schläge und Tritte durch Dämpfungseffekte
          des Gewebes besser weg~\cite[99]{Fraunschiel2008}.
    \item Geschwungene Schläge und sind um den Faktor zwei energiereicher als
          gerade~\cite[100]{Fraunschiel2008}.
    \item Geschwungene Tritte sind um den Faktor 3,5 energiereicher als
          geschwungene Schläge und um den Faktor 7 energiereicher als gerade
          Schläge~\cite[100]{Fraunschiel2008}.
\end{itemize*}

% Zweite Arbeit
%
\section*{\thesisheading}

\thesis{Titel}{Evaluating Object-Oriented Software Metrics for Source Code
               Change Analysis -- A Study on Open Source Projects}
\thesis{AutorIn}{Andreas Mauczka}
\thesis{Betreuung}{Thomas Grechenig und Mario Bernhart}
\thesis{Preis}{nein}

% 3-5 Sätze zu jedem der folg. Punkte jeder Diplomarbeit

\minisec{Themengebiet} Das Themengebeit umfasst Software Metriken und deren
Validierung in Bezug auf Code Changes. Dabei werden ausschließlich Metriken
für objektorientierte Programmiersprachen verwendet. Datamining von
Versionsverwaltungssystemen spielt ebenfalls eine wichtige Rolle da aus diesen
die Daten für die weitere Analyse gewonnen werden.

\minisec{Problemstellung} Ziel der Arbeit ist eine Methode zu entwickeln
welche das Validieren von Software Metriken mittels Change Data aus
Versionsverwaltungssystemen ermöglicht. Dabei wurde insbesondere auf die Frage
eingegangen: Gibt es einen Zusammenhang zwischen Code Changes und Software
Metriken? Für die Konkreten Ziele wurden Hypothesen
definiert~\cite[Kap.~6.2,~S.~39]{Mauczka2008}.

\minisec{Verwendete Methoden} Als verwendete Methoden kommen Literatur
Recherche \& Zitieren und Implementieren zum Einsatz. Weiters wurde eine
Fallstudie mit Open Source Projekten durchgeführt. Für die Fallstudie wurden
100 Projekte von Sourceforge\footnote{Sourceforge:
\url{http://sourceforge.net}} anhand von zuvor festgelegten Kriterien
ausgewählt und mittels Metriken gemessen. Anschließend erfolgte eine
statistische Auswertung um die aufgestellten Hypothesen zu testen.

\minisec{Ergebnis} Als Ergebnis wird gezeigt, dass es teilweise einen
signifikanten Zusammenhang zwischen bestehenden Software Metriken und und Code
Changes gibt. Dabei korrelieren komplexere Metriken weniger stark mit Code
Changes. Allgemein kann auch gezeigt werden, dass Code Level Metriken besser
als Struktur Metriken mit Code Changes korrelieren. Der Versuch die Stabilität
von Code ebenfalls anhand der Metriken und Change Data zu untersuchen musste
aufgegeben werden~\cite[Kap.~8,~S.~83]{Mauczka2008}.


% Dritte Arbeit
%
\section*{\thesisheading}

\thesis{Titel}{J3DVN -- A Generic Framework for 3D Software Visualization}
\thesis{AutorIn}{Florian Breier}
\thesis{Betreuung}{Harald Gall und Jacek Ratzinger}
\thesis{Preis}{nein}

% 3-5 Sätze zu jedem der folg. Punkte jeder Diplomarbeit

\minisec{Themengebiet}

\minisec{Problemstellung}

\minisec{Verwendete Methoden}

\minisec{Ergebnis}


% section section_name (end)