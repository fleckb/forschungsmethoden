%!TEX root = report.tex

% Erste Arbeit
%
\section*{\thesisheading} % (fold)

\thesis{Titel}{Erhebung und Evaluierung von Akzeptanz und Nutzung der 
                zukünftigen elektronischen Patientenakte in Österreich}
\thesis{AutorIn}{René Baranyi}
\thesis{Betreuung}{Univ. Prof. Dipl.-Ing. Dr. Thomas Grenchenig}
\thesis{Preis}{nein}

% 3-5 Sätze zu jedem der folg. Punkte jeder Diplomarbeit

\minisec{Themengebiet} Das Themengebiet umfasst die Akzeptanz der geplanten
EPA, den gewünschten Anforderungen an diese, den sicherheitsrelevanten
Bedenken und den Erwartungen hinsichtlich unterschiedlicher Altersgruppen.

\minisec{Problemstellung} Die EPA wird derzeit in Österreich durch die
Bundesregierung in Auftrag gegeben und durch mehrere Firmen aus verschiedenen
Arbeitskreisen geplant. Das Ziel der Arbeit ist es aufzuzeigen, dass die
Enduser schon heute in den Planungsprozess mehr einbezogen werden müssen um zu
gewährleisten, dass die EPA effizient auf diese zugeschnitten ist.

\minisec{Verwendete Methoden} Als verwendete Methoden kommen eine
Literaturrecherche, Hypothesen und Umfragen zum Einsatz. Die Umfrage wurde
mit Hilfe eines eigens entwickelten Fragebogen durchgeführt. Anschließend
wurde zunächst der Fragebogen reevaluiert um eine bessere Auswertung der
Ergebnisse im Kontext zu ermöglichen.

\minisec{Ergebnis} Als Ergebnis der Arbeit zeigte sich, dass die befragten
Testpersonen sehr wohl konkrete Vorstellungen in Bezug auf die geplante
Einführung und Nutzung der EPA haben. Die entstehenden Kosten im Zusammenhang
mit der Entwicklung und Einführung der EPA wollen die Befragten jedoch nicht
selber tragen, betrachten aber die Investition von mehreren Millionen Euro als
sinnvoll. Wenn man die gesammelten Daten in Vergleich zu den geplanten
Maßnahmen stellt, zeigt die Arbeit sehr deutlich, dass die Einbeziehung der
Enduser in den Entwicklungsprozess als sinnvoll und erstrebenswert zu
betrachten ist.


% Zweite Arbeit
%
\section*{\thesisheading} % (fold)

\thesis{Titel}{Heimautomationssysteme Theorie und Praxis}
\thesis{AutorIn}{Hannes Brandstetter}
\thesis{Betreuung}{Ao. Univ. Prof. Dipl.-Ing Dr.techn. Wolfgang Kastner}
\thesis{Preis}{nein}

% 3-5 Sätze zu jedem der folg. Punkte jeder Diplomarbeit

\minisec{Themengebiet} Das Themengebiet umfasst die verschiedenen Arten von
Heimautomationssystemen. Die Kriterien welche ein HAS erfüllen muss wurden
genau eroiert und ein Vergleich zwischen der unterschiedlichen HAS bwzüglich
der Standards evaluiert. Weiters befasst sich die Arbeit auch mit der
Sicherheit, der Topologie und der Bedienbarkeit der Systeme.

\minisec{Problemstellung} Ziel der Arbeit ist es Kriterien zu ermitteln, die
zur individuellen Entscheidungsfindung für ein Heimautomatisierungssystem
relevant sein können. Im Rahmen der Arbeit ein Projekt vorgestellt, welches
eine kostengünstige Variante für eine Haussteuerung bietet. Im Projekt wurde
speziell darauf geachtet, dass die Systemkomponenten eine offene Architektur
anbieten, um für zukünftige Erweiterungen eine Schnittstelle zu bieten.

\minisec{Verwendete Methoden} Als verwendete Methoden kommen eine
Literaturrecherche, eine Modellierung sowie ein Experiment zum Einsatz. Die
Modellierung belief sich auf die Planung des Heimautomationssystems auf
Grundlage der in der Recherche gewonnenen Erkenntnisse. Zuletzt wurde
ebenfalls die Häufigkeit und die Art der verwendeten Befehle zur Steuerung des
Systems, in einem Experiment, erfasst.

\minisec{Ergebnis} Das Ergebnis der Arbeit war ein eigens zusammengestelltes
und aufgebautes Heimautomationssystem mit der Möglichkeit zur Auswertung von
Usereingaben. Weiters zeigte die Arbeit ebenfalls, dass momentan die
verschiedenen Systeme, welche am Markt sind, kaum miteinander kombinierbar
sind da es keinen einheitlichen Standard für HAS gibt.


% Dritte Arbeit
%
\section*{\thesisheading} % (fold)

\thesis{Titel}{Einsatz von Honeyclient Technologien zur Steigerung der Sicherheit im Universitärem Umfeld}
\thesis{AutorIn}{Christian Kekeiss}
\thesis{Betreuung}{Univ. Prof. Dipl.-Ing. Dr. Thomas Grenchenig}
\thesis{Preis}{nein}

% 3-5 Sätze zu jedem der folg. Punkte jeder Diplomarbeit

\minisec{Themengebiet} Das Themengebiet umfasst die Detektion von
client-seitigen Angriffen mit Hilfe von Honeyclients. Das Hauptthema mit dem
sich die Arbeit befasst ist die  Sicherheit in Firmennetzwerken bzw. in einem
universitären Netzwerk. Weiters wurden die verschiedenen Honeyclients
verglichen und deren Einsatz im universitären Umfeld evaluiert.

\minisec{Problemstellung} Heutzutage sind Computer Ziele von Cyberangriffen,
wobei wissentlich oder unwissentlich auch der User selber zum Täter werden
kann indem er oder sie verseuchte Internetseiten aufruft. Um diese Gefahren zu
minimieren bedient man sich Honeyclients. Ziel der Arbeit ist es zu
evaluieren, ob die theoretischen Unterschiede von Honeyclients, welche aus
der Literatur hervorgehen, auch in der Praxis im universitären Umfeld
bestehen.

\minisec{Verwendete Methoden} Als verwendete Methoden kommen eine
Literaturrecherche, Implementierungen, Benchmarking und Simulation zum
Einsatz. Zunächst wurden verschiedene Honeyclient Versionen auf ein System
installiert und anschließend einzeln hinsichtlich erwartetem und tatsächlichen
Verhalten untersucht. Das Benchmarking zeigte zuletzt auf, welche  Version die
effizienteste Lösung war.

\minisec{Ergebnis} Das Ergebnis der Arbeit war eine detaillierte Analyse der
verschiedenen Honeyclients im tatsächlichen Einsatz. Dabei ist man auch auf
unerwartet Ergebnisse gestoßen, wie z.B. das der Großteil der
Implementierungen ohne eigens programmierte Erweiterungen nicht für eine
automatisierte Analyse aller Internetseiten in einem Netzwerk geeignet ist.
Zuletzt lässt sich sagen, dass die Technik der Honeyclients noch nicht
vollständig ausgereift ist, jedoch eine sehr interessante Methode ist um
zukünftig die Sicherheit in Netzwerken zu erhöhen.


% section section_name (end)